\documentclass[10pt,a4paper]{moderncv/moderncv}

\moderncvstyle{classic}       % style options are 'casual' (default), 'classic', 'oldstyle' and 'banking'
\moderncvcolor{blue}          % color options 'blue' (default), 'orange', 'green', 'red', 'purple', 'grey' and 'black'

\usepackage{fontspec}
\setmainfont{Palatino}
\setmonofont{Monaco}
\setsansfont{Helvetica}

\usepackage[us]{datetime}
\usepackage{footmisc} % enabling footnotes.
\makeatletter
\def\blfootnote{\xdef\@thefnmark{}\@footnotetext}
\makeatother

\usepackage[scale=0.83]{geometry}  % adjust the page margins
\recomputelengths                             % required when changes are made to page layout lengths

% personal data
\firstname{\textbf{Pei}}
\familyname{\textbf{QING}}
\email{edward.qing@gmail.com}
% \email{edwardtoday@gmail.com}
\phone[mobile]{+86 18621110317}


\social[github]{edwardtoday}
\homepage{qingpei.me}

% \quote{``Do what you fear, and the death of fear is certain.''\\-- Anthony Robbins}

\nopagenumbers{}            % uncomment to suppress automatic page numbering for CVs longer than one page

\begin{document}
\maketitle

\section{Education}
\cventry{2011--2012}{M.Sc. in Software Technology}{The Hong Kong Polytechnic University}{Hong Kong}{GPA:3.95/4 \textbf{with distinction}}{}
\cventry{2006--2010}{B.Eng. in Computer Science and Technology}{Tsinghua University}{Beijing}{GPA:85/100}{}
\cventry{2007--2010}{BS in Economics}{Tsinghua University}{Beijing}{second major}{}

% \section{Master thesis}
% \cvline{title}{\emph{Title}}
% \cvline{supervisors}{Supervisors}
% \cvline{description}{\small Short thesis abstract}

\section{Experience}
\cventry{2012--present}{Research Assistant}{Biometrics Research Center, The Hong Kong Polytechnic University}{Hong Kong}{}
{
\begin{description}
	\item[Data Source:] Samples of pulse, electrocardiogram, tongue photo, front face photo and breath composition analysis results from both healthy and diabetes people.
	\item[Objective:] To discover if some disease (diabetes in this case) has strong correlation with multiple physical signs of human body. Build a diagnosis model bases on the correlation to improve accuracy and consistency of diagnoses.
	\item[Methodology:] Iterations of ``Feature Extraction $\rightarrow$ Fusion of Multiple Features $\rightarrow$ Feature Optimization $\rightarrow$ Machine Learning $\rightarrow$ Model Simplification'' process.
	\item[Current Status:]  We achieved an accuracy of 97.1\% of diabetes diagnosis. By contrast, the accuracies using a single source of physical signs were 65\% to 90\%. Thus we have lowered the misdiagnosis rate by roughly an order of magnitude.
\end{description}
}

\cventry{2010--2011}{Software Developer}{Virtuos Games}{Shanghai}{}
{
\begin{itemize}
	\item Architected and implemented a cross-platform game on both PC/Mac and iPhone/iPad. In charge of the graphics performance optimization.
	\item Designed synchronization tool with C++/Python to improve inter-department cooperation.
	\item Acted as communicator between Technical and Art department to smooth out interdepartmental tasks.
	\item Acted as technical leader of internal iOS project. My demonstration attracted 3 clients to sign contracts with Virtuos on new iOS projects.
\end{itemize}
}

\cventry{2009}{Intern Programmer}{Endress+Hauser}{Shanghai}{}
{
\begin{itemize}
	\item Designed 3D models with Modo to present previews of customized products.
	\item Proposed and designed a product customization software with dynamic help for each part.
\end{itemize}
}

\cventry{2008}{Volunteer, Assistant Officer}{International Broadcast Center, 29th Olympic Games}
{Beijing}{}
{
\begin{itemize}
	\item Collected needs from agents of different media. Contact officers from inside IBC to meet their needs, or turn to Director for request necessary resources.
	\item Administrated attendance system of the venue, sending daily report on venue operations status.
\end{itemize}
}

\cventry{2007-2009}{Vice Director, Director}{Liaison Department, Student Union of Dept., Tsinghua University}
{Beijing}{}
{
\begin{itemize}
	\item Coordinated actions on seeking sponsors for student activities.
	\item Raised money for freshman orientation party.
\end{itemize}
}

\section{Projects}

\cventry{2013}
{image-converter-for-kindle: Convert JPEG images to kindle screensaver size}
{Python}
{}{}
{
Resize, crop, rotate and optimize images for the e-ink screen on Kindle.
}

% \cventry{2013}
% {mymaxim: Show random maxim in my blog posts.}
% {Javascript}
% {}{}
% {
% A script that shows random maxims on my personal site.
% }
% 
\cventry{2012}
{hkputhesis: The Hong Kong Polytechnic University M.Sc. thesis latex template}
{LaTeX}
{}{}
{
No one has ever made such a template for HKPU thesis. While I was writing my M.Sc. dissertation, I wrote the template and open-sourced it in case someone needed them in the future.
}

\cventry{2011--2012}
{3D Palm-print Recognition}
{MATLAB}
{}{}
{
\begin{itemize}
	\item Achieved a world leading 98.7\% identification accuracy using 3D palm-print features. (Accuracies found in previous literature were less than 93\%.)
	\item 2X speedup (compared to searching sorted database) gained by utilizing 3D global feature index.
\end{itemize}
}

\cventry{2010}
{Real-time Parallel Decoding of Multi-view Video}
{C++}
{Adviser: Lifeng Sun}{\textit{Scoring 93/100, top 10\%}}
{
\begin{itemize}
	\item Involved in the scheduling algorithm design for dual-view to 8-view MVC video.
	\item Assisted to implement a MVC enc/decoding tool, focusing on the decoding.
	\item Designed and implemented a 3D MVC player on the NVIDIA 3D Vision platform.
	\item Adopted by CCTV to provide experimental 3D online broadcasting of the 2010 Asian Games.
\end{itemize}
}

% \cventry{2010}
% {MIPS CPU Simulator}
% {Java}
% {}{}
% {
% \begin{itemize}
% 	\item A simulator that reads assembly code and illustrates how a MIPS CPU would execute it.
% 	\item States of all registers and memory spaces can be shown.
% \end{itemize}
% }

\cventry{2009}
{Intelligent Video Processing}
{C++}
{}{}
{
\begin{itemize}
	\item Implemented mean shift, GrabCut as pre-processors in the image processing chain.
	\item Modified GrabCut by replacing user interaction with edge detection to achieve automatic object library creation.
	\item Gave presentations in weekly discussion on the following paper: ARDECO, Interactive Video Cutout, Photo Clip Art, Video Object Cut and Paste.
\end{itemize}
}

\cventry{2009}
{Ray-tracing Renderer and Mesh Simplification}
{C++}
{Computer Graphics course project}{ranked 6/90+ students}
{{
\begin{itemize}
	\item Implemented a C++ ray-tracing renderer with Phong model.
	\item Top 10\% by rendering speed in class.
	\item Implemented both vertex decimation and edge contraction algorithms to simplify a mesh to a customizable complexity.
	\item Provided real-time preview of simplification progress with OpenGL.
\end{itemize}}
}

\cventry{2009}
{PhoneMe: A cross-platform address book}
{Java}
{Software Engineering course project}{ranked 2/50+ students}
{
\begin{itemize}
	\item Drafted and maintained requirements, design and technical document.
	\item Committed 30\% code of the 16,000-line project.
	\item Allocated tasks to team members weekly and tracked daily progress and issues.
	\item Held project discussion weekly to minimize mis-communication costs among members.
	\item Worked as a team leader.
\end{itemize}
}

% \cventry{2009}
% {Simple FTP Client and Server}
% {C++}
% {}{}
% {
% Implemented a lite FTP server and a client with C++ socket.
% }

\cventry{2008}
{16-bit FPGA MIPS CPU}
{VHDL}
{}{}
{
\begin{itemize}
	\item Designed a five-step pipelining structure.
	\item Implemented a processor in VHDL.
	\item Drove several peripherals including audio and video output.
	\item Processor frequency can be up to 50MHz on Cyclone II FPGA.
	\item Worked as a team leader.	
\end{itemize}
}

% \cventry{2008}
% {Mine Sweeper on FPGA}
% {VHDL}
% {}{}
% {
% Implemented a mine sweeper game on FPGA platform with mouse, keyboard and VGA controllers included.
% }


% Publications from a BibTeX file without multibib
% to redefine the heading string ("Publications"): \renewcommand{\refname}{Articles}

\nocite{*}
\bibliographystyle{plain}
\bibliography{publications}       % 'publications' is the name of a BibTeX file

\section{Skills}
\cvitem{Programming}{Used in Working: MATLAB, C++, \LaTeX. Used in Personal Code: Python, Java, R}
\cvitem{Keynote}{Proficient in making readable and clear slides using Apple Keynote or Microsoft Powerpoint. Experienced in delivering presentations.}
\cvitem{Data Analyzing}{Good at computer aided data analysis tools such as Excel, GNU R. Able to write programs to process and interpret data.}
\cvitem{English}{Fluent in spoken and written English, CET-6 652, TOEFL 107}
\cvitem{Zhihu}{\url{zhihu.com/people/qingpei}, My questions and answers. (Chinese version of Quora)}

\section{Awards}
\cventry{2009}{Outstanding Student Leaders in Department of Computer Science and Technology}{Tsinghua University}{}{}{}
\cventry{2008}{Excellent Volunteer at International Broadcast Center (IBC)}{The Beijing Organizing Committee for the Games of the XXIX Olympiad}{}{}{}
\cventry{2006}{First Prize in Haiwen Cup Shanghai High-school English Competition}{Shanghai Municipal Education Commission}{}{}{}
\cventry{2004}{Second Prize in "21st Century Cup" National English Speaking Contest}{Foreign Language Teaching and Research Press}{}{}{}

\closesection{}                   % needed to renewcommands
\renewcommand{\listitemsymbol}{-} % change the symbol for lists

\vfill
\enlargethispage{\footskip}
\blfootnote{Last updated: \today}

% \clearpage
% %-----       letter       ---------------------------------------------------------
% % recipient data
% \recipient{Company Recruitment team}{Company, Inc.\\123 somestreet\\some city}
% \date{January 01, 1984}
% \opening{Dear Sir or Madam,}
% \closing{Yours faithfully,}
% \enclosure[Attached]{curriculum vit\ae{}}          % use an optional argument to use a string other than "Enclosure", or redefine \enclname
% \makelettertitle
% 
% Lorem ipsum dolor sit amet, consectetur adipiscing elit. Duis ullamcorper neque sit amet lectus facilisis sed luctus nisl iaculis. Vivamus at neque arcu, sed tempor quam. Curabitur pharetra tincidunt tincidunt. Morbi volutpat feugiat mauris, quis tempor neque vehicula volutpat. Duis tristique justo vel massa fermentum accumsan. Mauris ante elit, feugiat vestibulum tempor eget, eleifend ac ipsum. Donec scelerisque lobortis ipsum eu vestibulum. Pellentesque vel massa at felis accumsan rhoncus.
% 
% Suspendisse commodo, massa eu congue tincidunt, elit mauris pellentesque orci, cursus tempor odio nisl euismod augue. Aliquam adipiscing nibh ut odio sodales et pulvinar tortor laoreet. Mauris a accumsan ligula. Class aptent taciti sociosqu ad litora torquent per conubia nostra, per inceptos himenaeos. Suspendisse vulputate sem vehicula ipsum varius nec tempus dui dapibus. Phasellus et est urna, ut auctor erat. Sed tincidunt odio id odio aliquam mattis. Donec sapien nulla, feugiat eget adipiscing sit amet, lacinia ut dolor. Phasellus tincidunt, leo a fringilla consectetur, felis diam aliquam urna, vitae aliquet lectus orci nec velit. Vivamus dapibus varius blandit.
% 
% Duis sit amet magna ante, at sodales diam. Aenean consectetur porta risus et sagittis. Ut interdum, enim varius pellentesque tincidunt, magna libero sodales tortor, ut fermentum nunc metus a ante. Vivamus odio leo, tincidunt eu luctus ut, sollicitudin sit amet metus. Nunc sed orci lectus. Ut sodales magna sed velit volutpat sit amet pulvinar diam venenatis.
% 
% Albert Einstein discovered that $e=mc^2$ in 1905.
% 
% \[ e=\lim_{n \to \infty} \left(1+\frac{1}{n}\right)^n \]
% 
% \makeletterclosing

\end{document}
