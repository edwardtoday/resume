\documentclass[11pt,a4paper]{moderncv/moderncv}

\moderncvstyle{classic}       % style options are 'casual' (default), 'classic', 'oldstyle' and 'banking'
\moderncvcolor{blue}          % color options 'blue' (default), 'orange', 'green', 'red', 'purple', 'grey' and 'black'

\usepackage[BoldFont,SlantFont,CJKchecksingle]{xeCJK}
\xeCJKsetup{PunctStyle=hangmobanjiao}
\punctstyle{kaiming}
%\setCJKmainfont[BoldFont=Hiragino Sans GB]{Hiragino Sans GB}
\setCJKmainfont[BoldFont=Lantinghei SC Demibold]{XinGothic-ZhangYue W4}

\setCJKmonofont{Menlo}
\XeTeXlinebreaklocale "zh"
\XeTeXlinebreakskip = 0.5pt plus 1pt minus 0.1pt
\linespread{1.2} 						% 行间距
\setlength{\parskip}{0.3cm}

\usepackage{datetime}
\renewcommand{\today}{\number\year 年 \number\month 月 \number\day 日}
\usepackage{footmisc} % enabling footnotes.
\makeatletter
\def\blfootnote{\xdef\@thefnmark{}\@footnotetext}
\makeatother

\usepackage[scale=0.8]{geometry}  % adjust the page margins
\recomputelengths                             % required when changes are made to page layout lengths

% personal data
\firstname{\textbf{卿培}}
\familyname{}
\email{edward.qing@gmail.com}
% \email{edwardtoday@gmail.com}
\phone[mobile]{+86 18621110317}


\social[github]{edwardtoday}
\homepage{qingpei.me}

% \photo[64pt][0.35pt]{img/side.jpg}
% \quote{``Do what you fear, and the death of fear is certain.''\\-- Anthony Robbins}

% \nopagenumbers{}            % uncomment to suppress automatic page numbering for CVs longer than one page

\begin{document}
\maketitle

%\section{求职意向}
%\cvitem{开发工程师}{应用软件、系统前端/后端开发}

\section{教育经历}
\cventry{2011--2012}{软件科技专业 硕士}{香港理工大学}{香港}{学分绩:3.95/4 获“成绩优异”评价}{}
\cventry{2006--2010}{计算机科学与技术专业 学士}{清华大学}{北京}{学分绩:85/100}{}
\cventry{2007--2010}{经济学 学士}{清华大学}{北京}{二学位}{}

% \section{Master thesis}
% \cvline{title}{\emph{Title}}
% \cvline{supervisors}{Supervisors}
% \cvline{description}{\small Short thesis abstract}

\section{工作经历}
\cventry{2013--present}{软件工程师}{上海三思电子科技有限公司}{上海}{}
{
\begin{itemize}
	\item 独立完成LED显示屏、路灯控制系统的网络通信模块。对TCP/UDP及串口(RS-485)控制器通信提供统一的接口。通过异步设计,\textbf{吞吐量比旧版提升6倍}。
	\item 改进内部显示屏校正软件,从成像和视觉原理出发设计校正算法,系统实现比此前\textbf{快30倍}。
	\item 搭建内网GitLab服务器,将Git版本控制引入开发流程;同时促使同事用Redmine代替项目管理上的的口头沟通,优化组内协同工作流程和效率。
	\item 提议并迁移公司7台低占用率服务器至虚拟化方案,降低成本,方便管理。
\end{itemize}
}

\cventry{2012--2013}{助研}{人体生物特征识别研究中心}{香港理工大学计算学系}{}
{
\begin{description}
	\item[研究对象:] 健康与糖尿病人的脉搏、心电图、舌相、呼吸气体成分、脸部照片等数据。
	\item[研究目的:] 寻找疾病(以糖尿病为例)与多种体征之间有无强相关性,并利用存在的相关性建立诊断模型,提高诊断的准确性与一致性。
	\item[研究方法:] 多次迭代``特征提取$\rightarrow$特征融合$\rightarrow$特征优化$\rightarrow$机器学习$\rightarrow$模型简化''的过程。
	\item[当前成果:] 我们在400余人的采样上做到糖尿病诊断\textbf{97.1\%的准确率}。作为对比,仅利用单一类型数据特征建立的诊断模型,准确率在65\%到90\%不等。我们已经将模型的误诊率降低了大约一个数量级。
\end{description}
}

\cventry{2010--2011}{软件工程师}{上海维塔士软件科技公司}{上海}{}
{
\begin{itemize}
	\item 从零开始设计并实现了一个跨PC/Mac/iOS系统的航海游戏。独立完成帧率优化,将iPhone 4平台的帧率从引擎默认设定下的12fps提高至35fps。
	\item 内部协同工具(C++/Python),简化美工与技术部门交接流程,降低冲突解决成本。
	\item 协调程序与美工团队的工作,在全年三个项目中向客户按时交付每一个版本,无一延期。
	\item 负责iPhone/iPad实验项目开发,带领5人团队从零开始学习运用新技术按计划完成项目,吸引3家客户与公司合作该平台新项目。
\end{itemize}
}

%\cventry{2009}{实习程序员}{上海恩德斯豪斯自动化设备有限公司}{上海}{}
%{
%\begin{itemize}
%	\item 用Modo建立产品3D模型。
%	\item 提议并用Java设计实现了基于3D模型的设备选型系统,以利于市场部向客户展示。
%\end{itemize}
%}

\cventry{2008}{志愿者,事务助理}{第29届奥运会国际广播中心}
{北京}{}
{
\begin{itemize}
	\item 协调相应部门志愿者满足各国媒体的工作需求,或汇报部门主管寻求外部资源解决问题。
	\item 负责场馆工作人员考勤、汇总并发送每日运营情况报表。
\end{itemize}
}

%\cventry{2007-2009}{部长}{清华大学计算机系学生会外联部}
%{北京}{}
%{
%\begin{itemize}
%	\item 组织外联部成员联系在京系友并保持联络。
%	\item 向各方寻求赞助以覆盖系内学生活动、迎新晚会的支出。
%\end{itemize}
%}

\section{项目经验}
%\cventry{2013}
%{image-converter-for-kindle}
%{Python}
%{}{}
%{
%自动缩放、裁剪、旋转、优化图片,以便在Kindle的电子墨水屏幕上更好地显示。
%}

% \cventry{2013}
% {mymaxim}
% {Javascript}
% {}{}
% {
% 在我的个人网站上随机显示名言的脚本。
% }

\cventry{2012}
{hkputhesis: 香港理工大学计算学系硕士论文\LaTeX 模板}
{LaTeX}
{}{}
{
在此之前没有人做过这样一个模板。我写硕士论文期间,把自己写的模板整理出来并开源,方便将来有需要的人。
}

\cventry{2011--2012}
{3D掌纹识别}
{MATLAB}
{}{}
{
\begin{itemize}
	\item 用3D掌纹特征达到了世界领先的98.7\%的身份验证准确率。(此前的文献不到93\%。)
	\item 通过3D掌纹全局特征索引,将单个采样验证速度提高至2倍。
\end{itemize}
}

\cventry{2010}
{多视点视频的实时解码}
{C++}
{导师: 孙立峰}{\textit{分数 93/100, 前 10\%}}
{
\begin{itemize}
	\item 参与双目至8目多视点视频的解码器调度算法设计。
	\item 协助实现多视点视频编解码器,主要关注解码部分。
	\item 设计实现基于NVidia 3D Vision平台的立体视频播放器。
	\item 该项目被CCTV用于提供2010年亚运会的3D试验性网络视频转播。
\end{itemize}
}

\cventry{2010}
{MIPS CPU模拟器}
{Java}
{}{}
{
\begin{itemize}
	\item 读入汇编代码并展示MIPS CPU如何执行该代码。
	\item 所有寄存器和内存地址空间的状态可见。
\end{itemize}
}

%\cventry{2009}
%{视频智能处理}
%{C++}
%{}{}
%{
%\begin{itemize}
%	\item 实现meanshift、GrabCut等算法,用作图像预处理。
%	\item 修改GrabCut算法,将人机交互过程取消,用边缘检测代替其原有功能,实现全自动图像分割和物体数据库的建立。
%	\item 每周组会介绍一个算法,包括: ARDECO, Interactive Video Cutout, Photo Clip Art, Video Object Cut and Paste.
%\end{itemize}
%}

\cventry{2009}
{光线跟踪渲染器及模型网格简化系统}
{C++}
{计算机图形学课程项目}{排名 6/90+}
{
\begin{itemize}
	\item 用C++实现Phong模型光线跟踪渲染器。
	\item 渲染速度在课程全部学生的前10\%。
	\item 同时实现顶点删除和边折叠算法的网格简化代码,支持将网格简化至任意自定义复杂度。
%	\item 用OpenGL提供简化后网格的实时预览。
\end{itemize}
}

\cventry{2009}
{PhoneMe: 跨平台通讯录}
{Java}
{软件工程课程项目}{排名 2/50+}
{
\begin{itemize}
	\item 草拟并维护需求文档、设计文档和技术文档。
	\item 在16,000行项目代码中提交了其中30\%。
	\item 分配组内任务,并跟踪每日进度。
	\item 每周组织小组讨论,避免组员间的任务冲突与对项目细节的不同理解。
	\item 任3人团队的组长。
\end{itemize}
}

% \cventry{2009}
% {简单FTP服务器与客户端}
% {C++}
% {}{}
% {
% Implemented a lite FTP server and a client with C++ socket.
% }

%\cventry{2008}
%{16位MIPS指令集CPU的FPGA实现}
%{VHDL}
%{}{}
%{
%\begin{itemize}
%	\item 设计了五级流水线结构。
%	\item 用VHDL实现了一个MIPS CPU。
%	\item 驱动了包括VGA显示器和PS/2鼠标、键盘在内的外设。
%	\item 处理器主频在Cyclone II FPGA上可达50MHz.
%	\item 任3人团队的组长。
%\end{itemize}
%}

% \cventry{2008}
% {扫雷游戏的FPGA实现}
% {VHDL}
% {}{}
% {
% Implemented a mine sweeper game on FPGA platform with mouse, keyboard and VGA controllers included.
% }


% Publications from a BibTeX file without multibib
% to redefine the heading string ("Publications"): \renewcommand{\refname}{Articles}

\nocite{*}
\renewcommand{\refname}{发表论文}
\bibliographystyle{plain}
\bibliography{publications}       % 'publications' is the name of a BibTeX file


\section{工作技能}
\cvitem{计算机语言}{工作中使用: MATLAB, C++, \LaTeX; 个人项目使用: Python, Java, R}
\cvitem{演示文稿}{擅长用Apple Keynote、Powerpoint制作可读性高、重点突出的演示文稿}
\cvitem{数据分析}{熟练使用Excel制表,掌握计算机辅助数据分析,能够编写程序解决实际问题}
\cvitem{英语}{英语流利对话、读写,曾在国家级、省级英语竞赛获奖;CET-6 652分, TOEFL 107分}
% \cvitem{驾驶}{C1驾照,4年驾龄,无扣分记录、无事故记录}
% \cvitem{摄影摄像}{帮多位朋友拍摄证件照、肖像;2003年以来负责拍摄记录班级集体活动照片与纪录短片}
% \cventry{知乎}{\url{zhihu.com/people/qingpei}}{我在知乎上的回答}{}{}{}

\section{曾获奖项}
\cventry{2009}{计算机系优秀学生干部}{清华大学}{}{}{}
\cventry{2008}{国际广播中心优秀志愿者}{第29届奥运会北京奥组委}{}{}{}
\cventry{2006}{上海市海文杯英语竞赛一等奖}{上海市教育委员会}{}{}{}
\cventry{2004}{21世纪杯全国英语演讲比赛二等奖}{21世纪报社、外教社}{}{}{}

\closesection{}                   % needed to renewcommands
\renewcommand{\listitemsymbol}{-} % change the symbol for lists

\vfill
\enlargethispage{\footskip}
\blfootnote{最后更新: \today}

%\clearpage
%%-----       letter       ---------------------------------------------------------
%% recipient data
%\recipient{人力资源部}{XX有限公司\\一条街123号\\魔都}
%\date{\today}
%\opening{尊敬的X经理,}
%\closing{此致\\敬礼!}
%\enclosure[附件]{简历{}}          % use an optional argument to use a string other than "Enclosure", or redefine \enclname
%\makelettertitle
%
%本人從「香港日報」中得悉, 貴公司現徵求營業員一名,需有大專以上程度及三年同類工作經驗。本人對是項職位深感興趣,故特致函應徵。
%
%本人畢業於香港中文大學工商管理學系,深具商業方面的知識。而本人於畢業後亦曾於龍圖貿易國際集團任職達三年之久,將本人所學付諸實踐。因此就學業及工作經驗方面,本人均合乎 貴公司的入職要求。
%
%本人具有良好的數理及商業頭腦,曾於九八年度全港商機創業大賽中獲得第三名。本人為人謹慎、處事快及具責任感,相信定能勝任這項工作。
%
%本人素仰 貴公司信譽超卓,若能被 貴公司所錄用,實為本人之榮幸。現隨函附上個人履歷表一份,並希望 貴公司能給予接見機會,以更詳盡地提供本人的資料供 貴公司參考。
%
%\makeletterclosing

\end{document}
