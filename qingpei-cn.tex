\documentclass[11pt,a4paper]{moderncv/moderncv}

\moderncvstyle{classic}       % style options are 'casual' (default), 'classic', 'oldstyle' and 'banking'
\moderncvcolor{blue}          % color options 'blue' (default), 'orange', 'green', 'red', 'purple', 'grey' and 'black'

% \usepackage{xunicode, xltxtra}
% \XeTeXlinebreaklocale "zh"
% \widowpenalty=10000
% %\setmainfont[Mapping=tex-text]{文泉驿正黑}
% \usepackage{CJKutf8}
% \setmainfont[Mapping=tex-text]{Hiragino Sans GB}
% \setsansfont[Mapping=tex-text]{Hiragino Sans GB}
% \CJKtilde

\usepackage[BoldFont,SlantFont,CJKchecksingle]{xeCJK}
\punctstyle{kaiming}
\setCJKmainfont[BoldFont=Hiragino Sans GB]{Hiragino Sans GB}
\setCJKmonofont{Monaco}
\XeTeXlinebreaklocale "zh"
\XeTeXlinebreakskip = 0pt plus 1pt
\setlength{\parskip}{0.3cm}	

\usepackage[scale=0.8]{geometry}  % adjust the page margins
\recomputelengths                             % required when changes are made to page layout lengths

% personal data
\firstname{卿培}
\familyname{}
\email{edward.qing@gmail.com}
% \email{edwardtoday@gmail.com}
\phone[mobile]{+86 18621110317}


\social[github]{edwardtoday}
\homepage{qingpei.me}

\photo[64pt][0.5pt]{img/side.jpg}
% \quote{``Do what you fear, and the death of fear is certain.''\\-- Anthony Robbins}

\nopagenumbers{}            % uncomment to suppress automatic page numbering for CVs longer than one page

\begin{document}
\maketitle

\section{教育经历}
\cventry{2011--2012}{软件科技专业 硕士}{香港理工大学}{香港}{学分绩:3.95/4}{获“成绩优异”评价}
\cventry{2006--2010}{计算机科学与技术专业 学士}{清华大学}{北京}{学分绩:85/100}{}
\cventry{2007--2010}{经济学 学士}{清华大学}{北京}{二学位}{}

% \section{Master thesis}
% \cvline{title}{\emph{Title}}
% \cvline{supervisors}{Supervisors}
% \cvline{description}{\small Short thesis abstract}

\section{工作经历}
\cventry{2012至今}{助研}{人體生物特徵識別研究中心}{香港理工大学计算学系}{}
{
}

\cventry{2010--2011}{软件工程师}{上海维塔士软件科技公司}{上海}{}
{
\begin{itemize}
	\item 从零开始设计并实现了一个跨PC/Mac/iOS系统的航海游戏。独立完成帧率优化,将iPhone 4平台的帧率从引擎默认设定下的12fps提高至35fps。
	\item 用C++和Python打造内部协同工作工具,简化美工与技术部门交接工作的流程,降低解决冲突的成本。
\end{itemize}
}

\cventry{2009}{实习程序员}{上海恩德斯豪斯自动化设备有限公司}{上海}{}
{
\begin{itemize}
	\item 用Modo建立产品3D模型。
	\item 提议并用Java设计实现了基于3D模型的设备选型系统,以利于市场部向客户展示。
\end{itemize}
}

\cventry{2008}{志愿者,事务助理}{第29届奥运会国际广播中心}
{北京}{}
{
\begin{itemize}
	\item 负责场馆工作人员考勤、汇总并发送每日运营情况报表。
	\item 协助安排不同部门人员就餐时间,并尽量缩短临时访客的就餐等待时间。
\end{itemize}
}

\cventry{2007-2009}{部长}{清华大学计算机系学生会外联部}
{北京}{}
{
\begin{itemize}
	\item 协调部内成员联系市内企业。
	\item 向各方寻求赞助以覆盖系内学生活动、迎新晚会的支出。
\end{itemize}
}

\section{项目经验}
\cventry{2013}
{image-converter-for-kindle}
{Python}
{}{}
{
自动缩放、裁剪、旋转、优化图片,以便在Kindle的电子墨水屏幕上更好地显示。
}

\cventry{2013}
{mymaxim}
{Javascript}
{}{}
{
在我的个人网站上随机显示名言的脚本。
}

\cventry{2012}
{hkputhesis: 香港理工大学计算学系硕士论文\LaTeX 模板}
{LaTeX}
{}{}
{
在此之前没有人做过这样一个模板。我写硕士论文期间,把自己写的模板整理出来并开源,方便将来有需要的人。
}

\cventry{2011--2012}
{3D Palmprint Recognition}
{MATLAB}
{}{}
{
\begin{itemize}
	\item 用3D掌纹特征达到了前所未有的98.7\%的分类准确率。(此前的文献不到93\%。)
	\item 通过3D掌纹全局特征索引,将单个采样验证速度提高至2倍。
\end{itemize}
}

\cventry{2010}
{多视点视频的实时解码}
{C++}
{导师: 孙立峰}{\textit{毕设分数 93/100, 前 10\%}}
{
\begin{itemize}
	\item 参与双目至8目多视点视频的解码器调度算法设计。
	\item 协助实现多视点视频编解码器,主要关注解码部分。
	\item 设计实现基于NVidia 3D Vision平台的立体视频播放器。
	\item 该项目被CCTV用于提供2010年亚运会的3D试验性网络视频转播。
\end{itemize}
}

\cventry{2010}
{MIPS CPU模拟器}
{Java}
{}{}
{
\begin{itemize}
	\item 读入汇编代码并展示MIPS CPU如何执行该代码。
	\item 所有寄存器和内存地址空间的状态可见。
\end{itemize}
}

\cventry{2009}
{智能视频处理}
{C++}
{}{}
{
\begin{itemize}
	\item 实现meanshift、GrabCut等算法,用作图像预处理。
	\item 修改GrabCut算法,将人机交互过程取消,用边缘检测代替其原有功能,实现全自动图像分割和物体数据库的建立。
	\item 每周组会介绍一个算法,包括: ARDECO, Interactive Video Cutout, Photo Clip Art, Video Object Cut and Paste.
\end{itemize}
}

\cventry{2009}
{光线跟踪渲染器及模型网格简化系统}
{C++}
{计算机图形学课程项目}{排名 6/90+}
{
\begin{itemize}
	\item 用C++实现Phong模型光线跟踪渲染器。
	\item 渲染速度在课程全部学生的前10\%。
	\item 同时实现顶点删除和边折叠算法的网格简化代码,支持将网格简化至任意自定义复杂度。
	\item 用OpenGL提供简化后网格的实时预览。
\end{itemize}
}

\cventry{2009}
{PhoneMe: 跨平台通讯录}
{Java}
{软件工程课程项目}{排名 2/50+}
{
\begin{itemize}
	\item 草拟并维护需求文档、设计文档和技术文档。
	\item 在16,000行项目代码中提交了其中30\%。
	\item 分配组内任务,并跟踪每日进度。
	\item 每周组织小组讨论,避免组员间的任务冲突与对项目细节的不同理解。
	\item 任3人团队的组长。
\end{itemize}
}

% \cventry{2009}
% {Simple FTP Client and Server}
% {C++}
% {}{}
% {
% Implemented a lite FTP server and a client with C++ socket.
% }

\cventry{2008}
{16-bit FPGA MIPS CPU}
{VHDL}
{}{}
{
\begin{itemize}
	\item 设计了五级流水线结构。
	\item 用VHDL实现了一个MIPS CPU。
	\item 驱动了包括VGA显示器和PS/2鼠标、键盘在内的外设。
	\item 处理器主频在Cyclone II FPGA上可达50MHz.
	\item 任3人团队的组长。	
\end{itemize}
}

% \cventry{2008}
% {Mine Sweeper on FPGA}
% {VHDL}
% {}{}
% {
% Implemented a mine sweeper game on FPGA platform with mouse, keyboard and VGA controllers included.
% }


% Publications from a BibTeX file without multibib
% to redefine the heading string ("Publications"): \renewcommand{\refname}{Articles}

\nocite{*}
\renewcommand{\refname}{发表论文}
\bibliographystyle{plain}
\bibliography{publications}       % 'publications' is the name of a BibTeX file

\section{曾获奖项}
\cventry{2009}{计算机系优秀学生干部}{清华大学}{}{}{}
\cventry{2008}{国际广播中心优秀志愿者}{第29届奥运会北京奥组委}{}{}{}
\cventry{2006}{上海市海文杯英语竞赛一等奖}{上海市教育委员会}{}{}{}


\section{掌握技能}
\cventry{程序设计}{Python = MATLAB = Java > C++ > R}{}{}{}{}
\cventry{英语}{CET-6 652, TOEFL 107}{英语流利读写}{}{}{}
% \cventry{Zhihu}{\url{zhihu.com/people/qingpei}}{My questions and answers. (Chinese version of Quora)}{}{}{}

\closesection{}                   % needed to renewcommands
\renewcommand{\listitemsymbol}{-} % change the symbol for lists

\end{document}
