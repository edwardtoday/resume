\documentclass[10pt,a4paper]{moderncv/moderncv}

\moderncvstyle{classic}       % style options are 'casual' (default), 'classic', 'oldstyle' and 'banking'
\moderncvcolor{blue}          % color options 'blue' (default), 'orange', 'green', 'red', 'purple', 'grey' and 'black'

% \usepackage{xunicode, xltxtra}
% \XeTeXlinebreaklocale "zh"
% \widowpenalty=10000
% %\setmainfont[Mapping=tex-text]{文泉驿正黑}
% \usepackage{CJKutf8}
% \setmainfont[Mapping=tex-text]{Hiragino Sans GB}
% \setsansfont[Mapping=tex-text]{Hiragino Sans GB}
% \CJKtilde

\usepackage[BoldFont,SlantFont,CJKchecksingle]{xeCJK}
\punctstyle{kaiming}
\setCJKmainfont[BoldFont=FZLanTingHeiS-B-GB]{FZLanTingHeiS-R-GB}
% \setCJKmainfont[BoldFont=Hiragino Sans GB]{Hiragino Sans GB}
\setCJKmonofont{Monaco}
\XeTeXlinebreaklocale "zh"
\XeTeXlinebreakskip = 0pt plus 1pt
\setlength{\parskip}{0.3cm}	

\usepackage[scale=0.8]{geometry}  % adjust the page margins
\recomputelengths                             % required when changes are made to page layout lengths

% personal data
\firstname{卿培}
\familyname{}
\email{edward.qing@gmail.com}
% \email{edwardtoday@gmail.com}
\phone[mobile]{+86 18621110317}


\social[github]{edwardtoday}
\homepage{qingpei.me}

\photo[64pt][0.35pt]{img/side.jpg}
% \quote{``Do what you fear, and the death of fear is certain.''\\-- Anthony Robbins}

\nopagenumbers{}            % uncomment to suppress automatic page numbering for CVs longer than one page

\begin{document}
\maketitle

\section{求职意向}
\cvitem{开发工程师}{应用软件、系统前端/后端开发}

\section{教育经历}
\cventry{2011--2012}{软件科技专业 硕士}{香港理工大学}{香港}{学分绩:3.95/4 获“成绩优异”评价}{}
\cventry{2006--2010}{计算机科学与技术专业 学士}{清华大学}{北京}{学分绩:85/100}{}
\cventry{2007--2010}{经济学 学士}{清华大学}{北京}{二学位}{}

% \section{Master thesis}
% \cvline{title}{\emph{Title}}
% \cvline{supervisors}{Supervisors}
% \cvline{description}{\small Short thesis abstract}

\section{工作经历}
\cventry{2012至今}{助研}{人体生物特征识别研究中心}{香港理工大学计算学系}{}
{
\begin{description}
	\item[研究对象:] 健康与糖尿病人的脉搏采样、心电图、舌头图像、呼吸气体成分测试结果、脸部照片等多种数据。
	\item[研究目的:] 寻找疾病(以糖尿病为例)与多种体征之间有无强相关性,并利用存在的相关性建立诊断模型,提高诊断的准确性与一致性。
	\item[研究方法:] 多次迭代``特征提取$\rightarrow$多种特征融合$\rightarrow$特征优化$\rightarrow$机器学习$\rightarrow$模型简化''的过程。
	\item[当前成果:] 我们在400余人的采样上做到糖尿病诊断97.1\%的准确率。作为对比,仅利用单一类型数据特征建立的诊断模型,准确率在65\%到90\%不等。我们已经将模型的误诊率降低了大约一个数量级。
\end{description}
}

\cventry{2010--2011}{软件工程师}{上海维塔士软件科技公司}{上海}{}
{
\begin{itemize}
	\item 从零开始设计并实现了一个跨PC/Mac/iOS系统的航海游戏。独立完成帧率优化,将iPhone 4平台的帧率从引擎默认设定下的12fps提高至35fps
	\item 用C++和Python打造内部协同工作工具,简化美工与技术部门交接工作的流程,降低解决冲突的成本
	\item 协调程序团队与美工团队的工作,帮助团队在全年项目中向客户按时交付每一个版本,无一推迟
	\item 担任公司iPhone/iPad实验项目开发团队负责人,带领5人团队从零开始学习运用新技术按计划完成项目,吸引3家客户与公司合作该平台新项目
\end{itemize}
}

\cventry{2009}{实习程序员}{上海恩德斯豪斯自动化设备有限公司}{上海}{}
{
\begin{itemize}
	\item 用Modo建立产品3D模型。
	\item 提议并用Java设计实现了基于3D模型的设备选型系统,以利于市场部向客户展示。
\end{itemize}
}

\cventry{2008}{志愿者,事务助理}{第29届奥运会国际广播中心}
{北京}{}
{
\begin{itemize}
	\item 协调相应部门志愿者满足各国媒体的工作需求,或汇报部门主管寻求外部资源解决问题
	\item 负责场馆工作人员考勤、汇总并发送每日运营情况报表
\end{itemize}
}

\cventry{2007-2009}{部长}{清华大学计算机系学生会外联部}
{北京}{}
{
\begin{itemize}
	\item 组织外联部成员联系在京系友并保持联络
	\item 向各方寻求赞助以覆盖系内学生活动、迎新晚会的支出
\end{itemize}
}

\section{项目经验}
\cventry{2013}
{image-converter-for-kindle}
{Python}
{}{}
{
自动缩放、裁剪、旋转、优化图片,以便在Kindle的电子墨水屏幕上更好地显示。
}

% \cventry{2013}
% {mymaxim}
% {Javascript}
% {}{}
% {
% 在我的个人网站上随机显示名言的脚本。
% }

\cventry{2012}
{hkputhesis: 香港理工大学计算学系硕士论文\LaTeX 模板}
{LaTeX}
{}{}
{
在此之前没有人做过这样一个模板。我写硕士论文期间,把自己写的模板整理出来并开源,方便将来有需要的人。
}

\cventry{2011--2012}
{3D掌纹识别}
{MATLAB}
{}{}
{
\begin{itemize}
	\item 用3D掌纹特征达到了世界领先的98.7\%的身份验证准确率。(此前的文献不到93\%。)
	\item 通过3D掌纹全局特征索引,将单个采样验证速度提高至2倍。
\end{itemize}
}

\cventry{2010}
{多视点视频的实时解码}
{C++}
{导师: 孙立峰}{\textit{分数 93/100, 前 10\%}}
{
\begin{itemize}
	\item 参与双目至8目多视点视频的解码器调度算法设计。
	\item 协助实现多视点视频编解码器,主要关注解码部分。
	\item 设计实现基于NVidia 3D Vision平台的立体视频播放器。
	\item 该项目被CCTV用于提供2010年亚运会的3D试验性网络视频转播。
\end{itemize}
}

\cventry{2010}
{MIPS CPU模拟器}
{Java}
{}{}
{
\begin{itemize}
	\item 读入汇编代码并展示MIPS CPU如何执行该代码。
	\item 所有寄存器和内存地址空间的状态可见。
\end{itemize}
}

\cventry{2009}
{智能视频处理}
{C++}
{}{}
{
\begin{itemize}
	\item 实现meanshift、GrabCut等算法,用作图像预处理。
	\item 修改GrabCut算法,将人机交互过程取消,用边缘检测代替其原有功能,实现全自动图像分割和物体数据库的建立。
	\item 每周组会介绍一个算法,包括: ARDECO, Interactive Video Cutout, Photo Clip Art, Video Object Cut and Paste.
\end{itemize}
}

\cventry{2009}
{光线跟踪渲染器及模型网格简化系统}
{C++}
{计算机图形学课程项目}{排名 6/90+}
{
\begin{itemize}
	\item 用C++实现Phong模型光线跟踪渲染器。
	\item 渲染速度在课程全部学生的前10\%。
	\item 同时实现顶点删除和边折叠算法的网格简化代码,支持将网格简化至任意自定义复杂度。
	\item 用OpenGL提供简化后网格的实时预览。
\end{itemize}
}

\cventry{2009}
{PhoneMe: 跨平台通讯录}
{Java}
{软件工程课程项目}{排名 2/50+}
{
\begin{itemize}
	\item 草拟并维护需求文档、设计文档和技术文档。
	\item 在16,000行项目代码中提交了其中30\%。
	\item 分配组内任务,并跟踪每日进度。
	\item 每周组织小组讨论,避免组员间的任务冲突与对项目细节的不同理解。
	\item 任3人团队的组长。
\end{itemize}
}

% \cventry{2009}
% {简单FTP服务器与客户端}
% {C++}
% {}{}
% {
% Implemented a lite FTP server and a client with C++ socket.
% }

\cventry{2008}
{16位MIPS指令集CPU的FPGA实现}
{VHDL}
{}{}
{
\begin{itemize}
	\item 设计了五级流水线结构。
	\item 用VHDL实现了一个MIPS CPU。
	\item 驱动了包括VGA显示器和PS/2鼠标、键盘在内的外设。
	\item 处理器主频在Cyclone II FPGA上可达50MHz.
	\item 任3人团队的组长。	
\end{itemize}
}

% \cventry{2008}
% {扫雷游戏的FPGA实现}
% {VHDL}
% {}{}
% {
% Implemented a mine sweeper game on FPGA platform with mouse, keyboard and VGA controllers included.
% }


% Publications from a BibTeX file without multibib
% to redefine the heading string ("Publications"): \renewcommand{\refname}{Articles}

\nocite{*}
\renewcommand{\refname}{发表论文}
\bibliographystyle{plain}
\bibliography{publications}       % 'publications' is the name of a BibTeX file


\section{工作技能}
\cvitem{计算机语言}{工作中使用: MATLAB, C++, \LaTeX; 个人项目使用: Python, Java, R}
\cvitem{演示文稿}{擅长用Apple Keynote、Powerpoint制作可读性高、重点突出的演示文稿}
\cvitem{数据分析}{熟练使用Excel制表,掌握计算机辅助数据分析,能够编写程序解决实际问题}
\cvitem{英语}{英语流利对话、读写,曾在国家级、省级英语竞赛获奖;CET-6 652分, TOEFL 107分}
% \cvitem{驾驶}{C1驾照,4年驾龄,无扣分记录、无事故记录}
% \cvitem{摄影摄像}{帮多位朋友拍摄证件照、肖像;03年以来负责拍摄记录班级集体活动照片与纪录短片}
% \cvitem{会计}{目前在准备2014年注册会计师资格考试《会计》、《审计》、《财务成本管理》三个科目}
\cventry{知乎}{\url{zhihu.com/people/qingpei}}{我在知乎上的回答}{}{}{}

\section{曾获奖项}
\cventry{2009}{计算机系优秀学生干部}{清华大学}{}{}{}
\cventry{2008}{国际广播中心优秀志愿者}{第29届奥运会北京奥组委}{}{}{}
\cventry{2006}{上海市海文杯英语竞赛一等奖}{上海市教育委员会}{}{}{}
\cventry{2004}{21世纪杯全国英语演讲比赛二等奖}{21世纪报社、外教社}{}{}{}

\closesection{}                   % needed to renewcommands
\renewcommand{\listitemsymbol}{-} % change the symbol for lists

% \clearpage
% %-----       letter       ---------------------------------------------------------
% % recipient data
% \recipient{Company Recruitment team}{Company, Inc.\\123 somestreet\\some city}
% \date{January 01, 1984}
% \opening{Dear Sir or Madam,}
% \closing{Yours faithfully,}
% \enclosure[Attached]{curriculum vit\ae{}}          % use an optional argument to use a string other than "Enclosure", or redefine \enclname
% \makelettertitle
% 
% Lorem ipsum dolor sit amet, consectetur adipiscing elit. Duis ullamcorper neque sit amet lectus facilisis sed luctus nisl iaculis. Vivamus at neque arcu, sed tempor quam. Curabitur pharetra tincidunt tincidunt. Morbi volutpat feugiat mauris, quis tempor neque vehicula volutpat. Duis tristique justo vel massa fermentum accumsan. Mauris ante elit, feugiat vestibulum tempor eget, eleifend ac ipsum. Donec scelerisque lobortis ipsum eu vestibulum. Pellentesque vel massa at felis accumsan rhoncus.
% 
% Suspendisse commodo, massa eu congue tincidunt, elit mauris pellentesque orci, cursus tempor odio nisl euismod augue. Aliquam adipiscing nibh ut odio sodales et pulvinar tortor laoreet. Mauris a accumsan ligula. Class aptent taciti sociosqu ad litora torquent per conubia nostra, per inceptos himenaeos. Suspendisse vulputate sem vehicula ipsum varius nec tempus dui dapibus. Phasellus et est urna, ut auctor erat. Sed tincidunt odio id odio aliquam mattis. Donec sapien nulla, feugiat eget adipiscing sit amet, lacinia ut dolor. Phasellus tincidunt, leo a fringilla consectetur, felis diam aliquam urna, vitae aliquet lectus orci nec velit. Vivamus dapibus varius blandit.
% 
% Duis sit amet magna ante, at sodales diam. Aenean consectetur porta risus et sagittis. Ut interdum, enim varius pellentesque tincidunt, magna libero sodales tortor, ut fermentum nunc metus a ante. Vivamus odio leo, tincidunt eu luctus ut, sollicitudin sit amet metus. Nunc sed orci lectus. Ut sodales magna sed velit volutpat sit amet pulvinar diam venenatis.
% 
% Albert Einstein discovered that $e=mc^2$ in 1905.
% 
% \[ e=\lim_{n \to \infty} \left(1+\frac{1}{n}\right)^n \]
% 
% \makeletterclosing

\end{document}
